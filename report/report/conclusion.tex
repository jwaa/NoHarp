\section{Conclusion}
The keyboard and mouse surpasses the LEAP-motion in every way. The keyboard-mouse combination is more efficient, faster and is perceived as more useable then the LEAP-motion. At least in the environment build for this experiment. The most important explanation for the big difference in all cases are the limitations in the hard- and software of the LEAP-motion. These technical shortcomings are partly caused by the intrinsic properties of the LEAP-motion. Another big part is our implementation. Take for example the range which was too limited for implementation but on the other hand the LEAP-motion was not designed for such wide operations. This lead to more mistakes in the classification of gestures which accounts for a big portion of the drop in efficiency \& speed and the reported irregularities while using the LEAP-motion. Furthermore participants reported fatigue while using the LEAP-motion. Although participant reported to be frustrated at certain points they still viewed the LEAP-motion as fun and full of potential.  

In other words the LEAP-motion is not suitable for a building blocks environment due to technical limitations. This can be extended to all kinds of applications where a wide range is needed and where users have to work with their arms for long periods of time. This does not mean that the LEAP-motion can be ruled out entirely. Participants still believe it could work in other situations.
