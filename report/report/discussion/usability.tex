\subsection{Usability}
Usability is being reviewed by using information gathered with the questionnaires. Two constructs were built from the multiple choice questions: \textsc{system usefulness} and \textsc{interface quality}. Furthermore some information can be extracted from the open question. 

A strong main effect of interaction device was found on the perceived system usefulness and interface quality. The keyboard-mouse combination scored significantly higher than the LEAP-motion. This means that the users experienced the keyboard and mouse the be more usable than the LEAP-motion independent of the order of experiment. 

The open questions give some insight why this is the case. Corresponding with the behavioral data the participants performed poorly using the LEAP-motion. They reported the tasks to be difficult to near impossible when using the LEAP-motion. In general the task objective and the concept of the interaction (how to operate the mouse/LEAP-motion) was clear. The poor efficiency and speed of the LEAP-motion resulted in poor usability. 

Other noteworthy aspects were the physical effort needed to use the LEAP-motion and fun-factor. At several occasions in the questionnaire participants reported that physical efforts needed to operate the LEAP-motion was too high. Besides all negative responses on the LEAP-motion the fun factor remained.

All in all participants saw potential in the LEAP-motion. Users would appreciate it better if the efficiency and the speed would improve and when it the setting is not work related e.g. in gaming. 
