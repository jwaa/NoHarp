\subsection{Design and Analysis}
The data had a multivariate within-subject design. A Pearson correlation analysis determined that \textsc{average speed}, \textsc{distance traveled} and \textsc{block repositioning} were very strongly correlated with keyboard as interaction device $(r \geq .771, n = 16, p \leq .000465)$. This was slightly different with the Leap motion as interaction device where respectively \textsc{average speed} with \textsc{distance traveled} and \textsc{block repositioning} with \textsc{block rotations} were strongly correlated $(r  \geq .557, n = 16, p \leq .025)$. Furthermore with both devices the \textsc{overal satisfaction} hightly correlated with \textsc{system usefulness} and \textsc{interface quality}$(r  \geq .502, n = 16, p \leq .047)$. \textsc{distance traveled} was arbitrarily left out of the analysis to reduce redundancy.  \textsc{Overal satisfaction} was also left out of the analysis, because it was measured by just one question and that overal quality could also had been measured by combining \textsc{system usefulness} and \textsc{interface quality}. 

A Shapiro-Wilk test on all remaining independent variables showed that almost all variables were not normally distributed $(W(16) \leq .864, p \leq .022 )$. There exceptions were \textsc{accuracy} for the Leap motion () and \textsc{interface quality} in general. For consistency all analysis were performed with non-parametric tests.

The results were analyzed with separate univariate Wilcoxon signed-ranked tests. To counter an increase of type I errors due to multiple testing the Bonferroni-correction was applied~\cite{dunn61}. The new significance level $\alpha$ was set on $00625$.
