\subsection{Design and Analysis}
The data had a multivariate within-subject design. A Pearson correlation analysis determined that \textsc{average speed}, \textsc{distance traveled} and \textsc{block repositioning} were very strongly correlated with keyboard as interaction device $(r \geq .771, n = 16, p \leq .000465)$. This was slightly different with the Leap motion as interaction device where respectively \textsc{average speed} with \textsc{distance traveled} and \textsc{block repositioning} with \textsc{block rotations} were strongly correlated $(r  \geq .557, n = 16, p \leq .025)$. Furthermore with both devices the \textsc{overalll satisfaction} hightly correlated with \textsc{system usefulness} and \textsc{interface quality}$(r  \geq .502, n = 16, p \leq .047)$. \textsc{distance traveled} was arbitrarily left out of the analysis to reduce redundancy. \textsc{Overalll satisfaction} was measured in this case by just one question (Q12) while this measure could also be constructed out of \textsc{system usefulness} and \textsc{interface quality}. Therefore \textsc{overalll satisfaction} was left out.

A Shapiro-Wilk test on all remaining independent variables showed that almost all variables were not normally distributed $(W(16) \leq .864, p \leq .022 )$\footnote{For only the significant results.}. There exceptions were \textsc{accuracy} for the Leap motion $(W(16) = .936, p = .307)$ and \textsc{interface quality} in general $(W(16) \geq .890, p \geq .055)$. For consistency all analysis were performed with non-parametric tests.

The results were analyzed with separate univariate Wilcoxon signed-ranked tests. To counter an increase of type I errors due to multiple testing the Bonferroni-correction was applied\cite{dunn61}. The new significance level $\alpha$ was set on $00625$.
