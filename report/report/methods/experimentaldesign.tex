\subsection{Experimental design (Janne)}
\subsubsection{Participants}
Sixteen people, ??? males and ??? females, participated in the experiment (meag age = ????, SD = ???). 
\subsubsection{Task and procedure}
The experiment consisted of two sessions: a LEAP-motion session (in which participants used the LEAP motion interface) and a keyboard \& mouse sessions (in which participants used the keyboard & mouse interface). The order of the sessions was counterbalanced.

Before the experiment participants received instructions about the task they were going to perform. Before every session participants received instructions about how to use the interface: which actions are possible, how to perform the actions (possible gestures, which buttons to use). After these instructions participants got 3 minutes to explore the interface that they were going to use in the upcoming session. 

Both sessions consisted of 3 trials. In each trial the participant had to replicate a target model (see figure ??).  In the LEAP motion session participants had to replicate the target models using gestures, while in the keyboard & mouse session they had to do this by using the keyboard and the mouse.

After the session participants filled in a Post-Study System Usability Questionnaire about the interface they used in that session (see Appendix A). So that every participant after the experiment has filled in the usability questionnaire for each interface.

For possible follow-up analyses the participants were, after the two session, asked to fill in a questionnaire about their preferences, background and previous experience and/or expertise with the two interfaces (see Appendix B). 
\newline
Uitleggen over tijdlimiet, printscreeens van schermen en dan uitleggen hoe de trials precies gingen etc
Moeten we ergens misschien een HCI cycle inbouwen?

\subsection{Design and measurements}
The experiment was done and analyzed using a within-subject design with as independent variable Interface (LEAP motion, keyboard & mouse) and dependent variables: usability, accuracy, efficiency and speed. 

Usability was measured using the Post-Study System Usability Questionnaire (PSSUQ) questionnaire [1]. Items are displayed with seven-point graphic scales with on the end points the terms “Strongly agree” for 1 and “Strongly disagree” for 7, and a “Not applicable” (N/A) point next to the scale. The PSSUQ was adapted to better fit with our system. questions 9-15 (measuring Information Quality) were excluded considering that our interfaces did not differ with respect to information provided. Furthermore, help or error messaging was neither used or necessary in our system. The adapted PSSUQ is attached in Appendix A. 

This questionnaire now measures three system qualities: Overall quality, System Usefulness and Interface quality.  Different items on the questionnaire respond to different system qualities:
\begin{itemize}
	\item Overall: Average the responses to Items 1 through 12.
	\item System Usefulness: Average the responses to Items 1 through 8.
    \item Interface Quality: Average the responses to Items 9 through 11.
\end{itemize}
\newline


