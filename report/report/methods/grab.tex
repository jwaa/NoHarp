\subsubsection{Block manipulation}
For both interfaces a way to move a block from one point to another point had to be implemented. This function can be split into three subfunctions. Picking up a block, moving a block, releasing the block. First will be discussed how this was implemented in the case of the keyboard and mouse interface. After that the implementation for the LEAP motion is discussed.

\paragraph{Keyboard and mouse}
In the keyboard and mouse interface, moving the block was solitary done with the mouse. To pick up a block the mouse button had to be clicked and held. As soon as the button was  released, the block was also released. To move the block the mouse could be moved. However, to avoid flattening the environment to a 2D environment the movement of the mouse only determines the position in the 2D grid shown. To get the block higher, the scroll wheel of the mouse was used. The reason for doing this, is to make the comparison with the LEAP motion interface more fair. Because in the LEAP motion interface, there are also three dimensions in which the hand has to move.

\paragraph{LEAP motion}
It was tried to keep the LEAP motion gesture as close to real life movements as possible. Therefore, the grabbing gesture had to be performed by closing a hand when it was shown above a block. The block was then shown inside the hand. While the hand remained closed the block would stay in the hand and move along with it. When the hand was opened the block would fall down at the position the hand was at that moment.

To indicate a grabbing gesture the function \texttt{Hand.scaleFactor(Frame sinceFrame)} from the LEAP motion SDK is used. This function looks at the scaling of the hand with all the fingers. When clenching the fist, the fingers get closer to the hand and therefore the scaling gets smaller, indicated by values $<1.0$. So for every frame, the program looks at the previous frame and finds a value of the scaling. When this value is lower than a certain threshold (in our case $<0.978$) it is saved. When the scaling gets lower for at least four frames, the system recognizes a grabbing gesture. When there are there are three or more frames in a row where the scaling is the same or bigger. The memory gets reset and the grabbing has to start over again.

After picking up it was easy to move the block, because the block would just follow the hand with which it was picked-up. Releasing was done in a similar manner as grabbing. When the scaling factor between the hand and the same hand in the previous frame was $>1.01$ for at least two frames the block got released. All thresholds were found with empirically.

An added feature to the LEAP motion interface was the fact that the hand should be near the block but not necessarily  on it to pick up the block. When detecting a grab gesture, the system would try to find a block within a certain parameter.