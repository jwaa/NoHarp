\subsubsection{The Rotate Gestures}

The control of the circular grid through the Leap Motion is done by
swipe gestures. This choice was inspired by the numerous available
interaction patterns where swipe gestures are used to navigate through
some kind of interface. Also, a full hand swipe in a horizontal orientated
direction is similar to the movement one makes when spinning a plate
such as the circular grid. 

For these reasons a set of swipe gestures were implemented to rotate
the circular grid in both a horizontal and vertical direction (adjust
the angle at which the interface shows the grid). These gestures are
called the rotate gestures and are divided into two groups; swipe
gestures that move in a horizontal direction and those that move in
a vertical direction. The decision boundary lies at a $45$ degree
angle from the horizontal axis, any gestures beyond that angle are
classified as a vertical rotate gesture.

The straightforward implementation of these gestures based on the
swipe data presented by the Leap Motion and the used SDK caused several
problems. The first being that the SDK returned every swipe it detected
but not in a reliable way; the tags showing if this continuous gesture
just started, continued or ended in the current frame did not correlate
with the actual swipe movement performed. Therefore a single fluent
swipe performed by the user could contain several different swipes
according to the SDK. Another problem was that when the user made
an unrestricted swipe in open space there was always a small movement
back to a central position specific to the user. The SDK and Leap
Motion detected this movement as a swipe as well while not intended
as a rotate gesture. 

To solve these two problems a robust way of filtering the swipes received
from the SDK had to be created. This was done by creating two small
user profiles (one for each rotate gesture category) based on all
previous perceived rotate gestures started with some initial values.
This profile contained the average coordinate on the z-axis and x-axis
for respectively the horizontal and vertical rotate gestures. Each
category also contained the average speed of these gestures and the
average time between the rotate gestures. Each average was calculated
in a moving average manner as well as its moving standard deviation.
With these two values, and a static predefined sensitivity value,
a range was constructed where the respective feature should lie in
between to be accepted. If all features of a certain swipe were accepted,
the swipe was classified as a rotate swipe and was performed. 

The initial values of these user profiles was based on extensive trials
of three expert users using the rotate gestures. Their combined user
profiles was taken as these initial values. The sensitivity values
where also based on these trials to find their most optimal values.

The rotation of the circular grid based on the rotate gestures is
done in a momentum-decay manner; the velocity of the rotate gestures
contributes to the current rotation momentum of the circular grid
while at each frame this momentum was decreased by a certain decay
factor. This implementation was inspired on the various applications
where swipe is used to scroll the interface but is not immediately
stopped when no swipe is detected and where several swipes in quick
succession cause an increase in scrolling speed. 

The decay value and rotate gesture contribution (scaled by a sensitivity
value) differed for the vertical and horizontal rotate gestures. The
vertical rotate gesture had a larger decay value and large swipe velocities
did not affected the momentum as much (low sensitivity values). This
was due to a desire of changing the angle of the circular grid in
a more precise manner, whereas the horizontal rotate gesture did not
had such a requirement.

The usage of these small user profiles and extensive testing of several
sensitivity values resulted in a robust implementation of the rotate
gestures that could adapt on the users own preferred way of swiping.
The addition of a momentum-decay rotation effect caused the circular
grid to move in a more natural and intuitive way compared to previous
experience the user could have with swipe based navigation.
